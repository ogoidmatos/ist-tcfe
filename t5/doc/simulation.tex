\section{Simulation Analysis}
\label{sec:simulation}
\tab In this section the steps that were made on Ngspice to simulate the bandpass filter will be described. Making use of an OP-AMP and the available components, our focus on the simulation was finding the best values to simulate a bandpass filter with the required characteristics. Said characteristics were the central frequency, which needed to be as close to 1000 Hz as possible, the gain which should be around 40 dB and the output/input impedances. 

Starting with the circuit, a circuit was designed using the components available. Then, making use of the available components, different resistors and capacitors in order to reach the requirements of the bandpass filter. The final results are presented in the following table:

\begin{table}[H]
    \centering
    \begin{tabular}{|l|r|}
    \hline
    {\bf Name} & {\bf Value} \\ \hline
   \input{../sim/gain_tab.tex}
    \end{tabular}
    \caption{Absolute values for input and output impedances of the circuit.}
    \label{tab:spice1}
\end{table}
 
As can be seen in Table~\ref{tab:spice1} this circuit filters both high and low frequencies, which was one of the goals of said circuit. Moreover, we can see that both the central frequencies and gain come close to the desired values, being this result of work made to optimize their values. 

Looking at the input impedance, where a high value was desired in order to get the voltages of Vin and In – as close as possible, we can see that this was accomplished in the Table~\ref{tab:spice2}.

\begin{table}[H]
    \centering
    \begin{tabular}{|l|r|}
    \hline
    {\bf Name} & {\bf Value} \\ \hline
   \input{../sim/zin_tab.tex}
   \input{../sim/zout_tab.tex}
    \end{tabular}
    \caption{Input and output impedances of the circuit.}
    \label{tab:spice2}
\end{table}

Moving on to the output impedance, this one should be as low as possible, since what is desired is a high output voltage.  To measure the output impedance, a voltage source was used, being its value relatively low as can be seen in the table above 

Computing the cost and the quality of the passband filter, a figure of merit was calculated, as can be seen in Table~\ref{tab:spice3}, allowing us to understand how efficient our BPF was. An effort was made to optimize its value, trying to make this BPF as good as possible in terms of quality/price.

\begin{table}[H]
    \centering
    \begin{tabular}{|l|r|}
    \hline
    {\bf Name} & {\bf Value} \\ \hline
   \input{../sim/merito_tab.tex}
    \end{tabular}
    \caption{Cost and figure of merit of the simulated circuit.}
    \label{tab:spice3}
\end{table}

\begin{figure}[H] \centering
    \includegraphics[width=0.5\linewidth]{spice1.eps}
    \caption{Simulated frequency response.}
    \label{fig:sim1}
\end{figure}
