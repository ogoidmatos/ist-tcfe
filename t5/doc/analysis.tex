\section{Theoretical analysis}

\label{sec:analysis}

\tab In this section, the circuit shown in Figure~\ref{fig:circuit} is analysed theoretically.

This circuit consists of an AMP-OP 741 and several resistances and capacitors. To analyse this circuit we considered the AMP-OP to be ideal (Zi = infinite and Zo = 0) and then we used nodal and mesh methods to derive the following equations:

\begin{equation}
\left|Z_{i}\right|=\left|Z_{C 1}+R 1 / / \infty\right|=\left|Z_{C 1}+R 1\right|
\end{equation}

\begin{equation}
\left|Z_{o}\right|=\left|Z_{C 2} / /(R 2+R 3 / / 0)\right|=\left|Z_{C 2} / / R 2\right|
\end{equation}

\begin{equation}
\left\{\begin{array}{l}
v_{-}=v_{+}=\frac{R 1}{R 1+Z_{C 1}} v_{i} \\
v_{A}=\left(1+\frac{R 3}{R 4}\right) v_{-} \\
v_{o}=\frac{Z_{C 2}}{Z_{C 2}+R 2} v_{A}
\end{array}\right.
\end{equation}

To obtain the central frequency, we calculated the average between the High Cut-off frequency (wh) and the Low Cut-off frequency (wl):
\begin{equation}
\omega_{0}=\sqrt{\omega_{L} * \omega_{H}}
\end{equation}

With these parameters we we're able to obtain a bandpass filter and by solving the previously presented equations we reached the following characteristics and frequency response graphic for the filter:


\begin{figure}[H] \centering
    \includegraphics[width=0.5\linewidth]{theo.eps}
    \caption{Theoretical frequency response.}
    \label{fig:theo1}
\end{figure}

\begin{table}[H]
    \centering
    \begin{tabular}{|l|r|}
    \hline
    {\bf Name} & {\bf Value} \\ \hline
   \input{../mat/gain.tex}
    \end{tabular}
    \caption{Voltage gain, bandwidth, central frequency, frequency and gain devitions.}
    \label{tab:theo1}
\end{table}


\begin{table}[H]
    \centering
    \begin{tabular}{|l|r|}
    \hline
    {\bf Name} & {\bf Value} \\ \hline
   \input{../mat/imp_octave.tex}
    \end{tabular}
    \caption{Absolute values for input and output impedances.}
    \label{tab:theo2}
\end{table}
%tabela
%grafico
