\section{Octave and Ngspice Comparison}
\label{sec:comparision}

\subsection{Analysis for t$<$0}
\begin{table}[H]
	\centering
  \begin{tabular}{|l|r|}
    \hline    
    {\bf Name} & {\bf Value [V]} \\ \hline
	\input{../mat/volt_tab1}
  \end{tabular}
  \begin{tabular}{|l|r|}
    \hline    
    {\bf Name} & {\bf Value [A]} \\ \hline
	\input{../mat/current_tab1}
  \end{tabular}
  \begin{tabular}{|l|r|}
    \hline    
    {\bf Name} & {\bf Value [A or V]} \\ \hline
    @cb[i] & 0.000000e+00\\ \hline
@ce[i] & 0.000000e+00\\ \hline
@q1[ib] & 7.022567e-05\\ \hline
@q1[ic] & 1.404513e-02\\ \hline
@q1[ie] & -1.41154e-02\\ \hline
@q1[is] & 5.765392e-12\\ \hline
@rc[i] & 1.411536e-02\\ \hline
@re[i] & 1.411536e-02\\ \hline
@rf[i] & 7.022567e-05\\ \hline
@rs[i] & 0.000000e+00\\ \hline
v(1) & 0.000000e+00\\ \hline
v(2) & 0.000000e+00\\ \hline
base & 2.254108e+00\\ \hline
coll & 5.765392e+00\\ \hline
emit & 1.411536e+00\\ \hline
vcc & 1.000000e+01\\ \hline

  \end{tabular}
  \caption{Voltages and currents obtained by node analysis in Octave on the left and by operating point in Ngspice on the right.}
    \label{tab:comp1}
\end{table}

\subsection{Natural solution for t$>$0}
\begin{table}[H]
  \centering
  \begin{tabular}{|l|r|}
    \hline    
    {\bf Name} & {\bf Value [A], [V], [Ohm]} \\ \hline
	\input{../mat/point2_tab}
  \end{tabular}
  \begin{tabular}{|l|r|}
    \hline    
    {\bf Name} & {\bf Value [A or V]} \\ \hline
    \input{../sim/op_2_tab}
  \end{tabular}
  \caption{On the left, results from Octave, and on the right Ngspice. These tables give us the current and voltage between nodes 6 and 8. All the currents and voltages in other nodes/branches are null.}
  \label{tab:comp2}
\end{table}
