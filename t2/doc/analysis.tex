\section{Theoretical Analysis}
\label{sec:analysis}

\tab In this section, the circuit shown in Figure~\ref{fig:circuit} is analysed
theoretically, through the method of Node Analysis.

\subsection{Node analysis for t<0}
\label{subsec:node}
\tab 
\begin{equation}
\begin{bmatrix}
  0 & 1 & 0 & 0 & 0 & 0 & 0 & 0 \\
  0 & G_1 & -G_1-G_3-G_2 & G_2 & G_3 & 0 & 0 & 0 \\
  0 & 0 & G_2+K_b & -G_2 & -K_b & 0 & 0 & 0 \\
  0 & 0 & 0 & 0 & 0 & 0 & -G_6-G_7 & G_7 \\
  0 & 0 & 0 & 0 & 1 & 0 & K_c*G_6 & -1 \\
  0 & 0 & -K_b & 0 & G_5+K_b & -G_5 & 0 & 0 \\
  1 & 0 & 0 & 0 & 0 & 0 & 0 & 0 \\
  0 & 0 & G_3 & 0 & -G_3-G_4-G_5 & G_5 & G_7 & -G_7
\end{bmatrix}
\begin{bmatrix}
  V_0 \\ V_1 \\ V_2 \\ V_3 \\ V_5 \\ V_6 \\ V_7 \\ V_8
\end{bmatrix}
=
\begin{bmatrix}
  V_s \\ 0 \\ 0 \\ 0 \\ 0 \\ 0 \\ 0 \\ 0
\end{bmatrix}
\end{equation}

The results obtained after implementing the KNL in Octave can be seen in Table~\ref{tab:octave_currents}:
%inserir tabela

%%%%
\begin{table}[h]
  \centering
  \begin{tabular}{|l|r|}
    \hline    
    {\bf Name} & {\bf Value [A]} \\ \hline
	\input{../mat/current_tab}
  \end{tabular}
  \caption{Currents obtained in the theoretical analysis in Octave.}
  \label{tab:octave_currents}
\end{table}
The currents derived from the results of the node analysis are:
%inserir tabela
\begin{table}[h]
  \centering
  \begin{tabular}{|l|r|}
    \hline    
    {\bf Name} & {\bf Value [A]} \\ \hline
	\input{../mat/current_tab}
  \end{tabular}
  \caption{Currents obtained in the theoretical analysis in Octave.}
  \label{tab:octave_currents}
\end{table}
%%%%

\subsection{Node Analysis for t>=0 (Natural solution)}
\label{subsec:node}
\tab Here, we made use of Node Analysis to determine the current that passed through the capacitor. By short-circuting the independent voltage source (V_s) and by swapping the capacitor with a voltage source (V_x = V_6 - V_8) we were able to calculate the current that flowed through the capacitor and the equivalent resistance. We short-circuted the independent voltage source because we are using the thevenin/norton theorems to calculate the resistence as seen by the capacitor and we swapped the capacitor with a voltage source as at t = 0 the capacitor begins discharging and it has a voltage of V_x (since it's voltage is at a peak). 
Node analysis matrix:
%Include matrix ----------
\begin{equation}
\begin{bmatrix}
  0 & 1 & 0 & 0 & 0 & 0 & 0 & 0 & 0 \\
  0 & G_1 & -G_1-G_3-G_2 & G_2 & G_3 & 0 & 0 & 0 & 0 \\
  0 & 0 & -G_2+K_b & G_2 & -K_b & 0 & 0 & 0 & 0 \\
  0 & 0 & 0 & 0 & 0 & 0 & -G_6-G_7 & G_7 & 0 \\
  0 & 0 & 0 & 0 & 1 & 0 & K_c*G_6 & -G_7 & 0 \\
  0 & 0 & K_b & 0 & G_5-K_b & -G_5 & 0 & 0 & 1 \\
  1 & 0 & 0 & 0 & 0 & 0 & 0 & 0 & 0 \\
  0 & 0 & G_1 & 0 & G_4 & 0 & G_6 & 0 & 0
\end{bmatrix}
\begin{bmatrix}
  V_0 \\ V_1 \\ V_2 \\ V_3 \\ V_5 \\ V_6 \\ V_7 \\ V_8 \\ I_x
\end{bmatrix}
=
\begin{bmatrix}
  0 \\ 0 \\ 0 \\ 0 \\ 0 \\ 0 \\ Vx \\ 0 \\ 0
\end{bmatrix}
\end{equation}



\begin{equation}

R_eq = (V_x / I_x)

\end{equation}

\begin{equation}

V_n = V_x * exp( -t / (R_eq * C))

\end{equation}
%inserir resultados

\par
With the results obtained before we ploted the natural solution in the following graphic:

%inserir grafico

%ponto 4

\subsection{Node Analysis for t>=0 (Forced solution)}

To determine the forced solution we used a phasor voltage source V_s = 1 V and then we applied the Node analysis (in which C was replaced by its impedance.

Node analysis matrix:
%Include matrix ----------
\begin{equation}
\begin{bmatrix}
  0 & 1 & 0 & 0 & 0 & 0 & 0 & 0 \\
  0 & G_1 & -G_1-G_3-G_2 & G_2 & G_3 & 0 & 0 & 0\\
  0 & 0 & -G_2+K_b & G_2 & -K_b & 0 & 0 & 0 \\
  0 & 0 & 0 & 0 & 0 & 0 & G_6-G_7 & G_7 \\
  0 & 0 & 0 & 0 & 1 & 0 & K_c*G_6 & -1 \\
  0 & 0 & K_b & 0 & -G_5+K_b & G_5+Z_c & 0 & -Z_c \\
  1 & 0 & 0 & 0 & 0 & 0 & 0 & 0 \\
  0 & -G_1 & 0 & 0 & G_4 & 0 & G_6 & 0
\end{bmatrix}
\begin{bmatrix}
  V_0 \\ V_1 \\ V_2 \\ V_3 \\ V_5 \\ V_6 \\ V_7 \\ V_8 
\end{bmatrix}
=
\begin{bmatrix}
  0 \\ exp(j) \\ 0 \\ 0 \\ 0 \\ 0 \\ 0 \\ 0 \\ 0
\end{bmatrix}
\end{equation}

%inserir tabelas amplitudes complexas DIOOOOOOOOOOOOOOOOOGOOOOOOOOOOOOOOOOOOOOOOOOOOOOOOOOOOOOOOOOOOOOOOOOOOOOOOOOOOOOOOOOOOOOOOOOOOOOOOOOOOOOOOOOOOOOOOOOOOOOOOOOOOOOOOOOOOOOOOOOOOOOOOOOOOOOOOOOOOOOOOOOOOOOOOOOOOOOOOOOOOOOOOOOOOOOOOOOOOOOOOOOOOOOOOOOOOOOOOOOOOOOOOOOOOOOOOOOOOOOOOOOOOOOOOOOOOOOOOOOOOOOOOOOO
% #GO BIG OR GO HOME


\subsection{Final Solution}

The final solution is: 

\begin {equation}
V_t = V_6 - V_8, for t<0
\par
V_t = V_n + abs(v_6) * sen (2 * pi * f * t + acos(Re(v_6) / abs(v_6)), for t>=0 
\end {equation}

\par
The plot of V_t and V_s is in the following graphic:

%inserir grafico diogo

\subsection{Frequency responses}
 In this subsection we present the frequency responses graphics.
 
 %inserir graficos

























