\section{Simulation Analysis}
\label{sec:simulation}
\tab

\subsection{Operating Point Analysis for t<0}
\label{subsec:op_analysis}

\tab While using Ngspice we encountered some problems with the use of current-controlled voltage sources. To overcome them, we introduced a dummy 0V voltage source between the resistors 6 and 7.
The voltage source $V_c$ depends on the current $I_c$, which is the current on $R_6$. However, since Ngspice could not take the current in $R_6$, we introduced the null voltage source, since the current there will be $I_c$, which is the current that $V_c$ depends on.

Table~\ref{tab:op} shows the simulated operating point results for the circuit
under analysis. Compared to the theoretical analysis results, one notices the
following differences: describe and explain the differences.

%referencias das tabelas 
Comparing the Tables \ref{tab:octave_currents} and \ref{tab:octave_voltages} with Table~\ref{tab:op}, we see that the difference between the different voltages and currents is apromixametly null.\par
%tabela
\begin{table}[h]
  \centering
  \begin{tabular}{|l|r|}
    \hline    
    {\bf Name} & {\bf Value [A or V]} \\ \hline
    @gib[i] & -2.60149e-04\\ \hline
@id[current] & 1.012649e-03\\ \hline
@rr1[i] & 2.482093e-04\\ \hline
@rr2[i] & -2.60149e-04\\ \hline
@rr3[i] & -1.19393e-05\\ \hline
@rr4[i] & 1.192184e-03\\ \hline
@rr5[i] & 1.272797e-03\\ \hline
@rr6[i] & 9.439751e-04\\ \hline
@rr7[i] & 9.439751e-04\\ \hline
v(1) & 7.053027e+00\\ \hline
v(2) & 6.795298e+00\\ \hline
v(3) & 6.264646e+00\\ \hline
v(4) & 1.949470e+00\\ \hline
v(5) & 6.831425e+00\\ \hline
v(6) & 1.080049e+01\\ \hline
v(7) & -9.54723e-01\\ \hline
v(9) & 0.000000e+00\\ \hline

  \end{tabular}
  \caption{Operating point. A variable preceded by @ is of type {\em current}
    and expressed in Ampere; other variables are of type {\it voltage} and expressed in
    Volt.}
  \label{tab:op}
\end{table}

\par
\subsection{Operating Point Analysis for t>0 (natural solution)}
By short-circuting the independent voltage source (V_s) and by swapping the capacitor with a voltage source (V_x = V_6 - V_8) we were able to calculate the current that flowed through the capacitor and the equivalent resistance. We short-circuted the independent voltage source because we are using the thevenin/norton theorems to calculate the resistence as seen by the capacitor and we swapped the capacitor with a voltage source as at t = 0 the capacitor begins discharging and it has a voltage of V_x (since it's voltage is at a peak).\par
The current source I_x is:

%inserir valor

\par and the R_eq is:

%inserir valor

\par 

The graphic for the natural solution, V_6(t), (using transient analysis) is the following:

%inserir grafico

\subsection{Operating Point Analysis for t>0 (natural and forced solution)}
\par

The graphic for the natural and forced responses on V_6(t):

%inserir grafico
\par
\subsection{Frequency response}
The graphics for the frequency response are:

%inserir graficos
\par
V_s and V_6 differ since V_6 is connected to a capacitor and therefore it will be the sum of a natural and a forced solution whilst V_s is just a voltage source. % ver se é preciso comentar aqui qlq cena mais (comparar c outro trabalho)

\section{Comparing Octave and Ngspice}

\subsection{Operating Point Analysis for t<0}
The difference between the results is minimal, as can be seen by comparing the table. The results are equal since the circuit is linear.

%inserir tabelas comparativas
\par

\subsection{Operating Point Analysis for t>0 (natural solution)}
The graphics are similar, as one would expect since the components are linear and therefore the circuit is linear.
%inserir graficos 
\par

\subsection{Operating Point Analysis for t>0 (forced solution)}

The graphics are similar, since the circuit is linear and we are using ideal voltage and current sources.

%inserir graficos

\subsection{Frequency response}

As one would expect the graphics are very similar and one can't find any significant discrepancies. This is the result of the circuit being linear

%inserir graficos

%rever comentarios aos graficos.

%atualizar indice





















































