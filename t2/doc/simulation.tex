\section{Simulation Analysis}
\label{sec:simulation}

\subsection{Operating Point Analysis for t$<$0}
\label{subsec:op_analysis}

\tab While using Ngspice we encountered some problems with the use of current-controlled voltage sources. To overcome them, we introduced a dummy 0V voltage source between the resistors 6 and 7.
The voltage source $V_c$ depends on the current $I_c$, which is the current on $R_6$. However, since Ngspice could not take the current in $R_6$, we introduced the null voltage source, since the current there will be $I_c$, which is the current that $V_c$ depends on.

Table~\ref{tab:op} shows the simulated operating point results for the circuit
under analysis. Compared to the theoretical analysis results, one notices the
following differences: describe and explain the differences.

%referencias das tabelas 
Comparing the Tables \ref{tab:octave_volt1} and \ref{tab:octave_currents1} with Table~\ref{tab:op}, we see that the difference between the different voltages and currents is apromixametly null.\par
%tabela
\begin{table}[H]
  \centering
  \begin{tabular}{|l|r|}
    \hline    
    {\bf Name} & {\bf Value [A or V]} \\ \hline
    @cb[i] & 0.000000e+00\\ \hline
@ce[i] & 0.000000e+00\\ \hline
@q1[ib] & 7.022567e-05\\ \hline
@q1[ic] & 1.404513e-02\\ \hline
@q1[ie] & -1.41154e-02\\ \hline
@q1[is] & 5.765392e-12\\ \hline
@rc[i] & 1.411536e-02\\ \hline
@re[i] & 1.411536e-02\\ \hline
@rf[i] & 7.022567e-05\\ \hline
@rs[i] & 0.000000e+00\\ \hline
v(1) & 0.000000e+00\\ \hline
v(2) & 0.000000e+00\\ \hline
base & 2.254108e+00\\ \hline
coll & 5.765392e+00\\ \hline
emit & 1.411536e+00\\ \hline
vcc & 1.000000e+01\\ \hline

  \end{tabular}
  \caption{Operating point. A variable preceded by @ is of type {\em current}
    and expressed in Ampere; other variables are of type {\it voltage} and expressed in
    Volt.}
  \label{tab:op}
\end{table}

\subsection{Operating Point Analysis for t$>$0 (natural solution)}
\tab By short-circuting the independent voltage source $V_s$ and by swapping the capacitor with a voltage source $V_x$ = $V_6$ - $V_8$ we were able to calculate the current that flowed through the capacitor and the equivalent resistance. We short-circuted the independent voltage source because we are using the thevenin/norton theorems to calculate the resistence as seen by the capacitor and we swapped the capacitor with a voltage source as at t = 0 the capacitor begins discharging and it has a voltage of $V_x$ (since it's voltage is at a peak).\par
The current source $I_x$ and voltage $V_6$ can be found in table~\ref{tab:op2}.

\begin{table}[H]
  \centering
  \begin{tabular}{|l|r|}
    \hline    
    {\bf Name} & {\bf Value [A or V]} \\ \hline
    \input{../sim/op_2_tab}
  \end{tabular}
  \caption{Operating point of point 2 of the simulation. A variable preceded by @ is of type {\em current}
    and expressed in Ampere; other variables are of type {\it voltage} and expressed in
    Volt.}
  \label{tab:op2}
\end{table}


The graphic for the natural solution, $V_6$(t), (using transient analysis) can be seen in Figure~\ref{fig:nat_sol}.

\begin{figure}[H] \centering
\includegraphics[width=0.6\linewidth]{../sim/sim_3.pdf}
\caption{Natural solution for capacitor.}
\label{fig:nat_sol}
\end{figure}
%inserir grafico

\subsection{Operating Point Analysis for t$>$0 (natural and forced solution)}

\tab The graphic for the natural and forced responses on $V_6$(t) and on $V_1$ (voltage source):

\begin{figure}[H] \centering
\includegraphics[width=0.6\linewidth]{../sim/sim_4.pdf}
\caption{Forced solution of voltage source and capacitor.}
\label{fig:nat_for}
\end{figure}
%inserir grafico
\subsection{Frequency response}
\tab The graphics for the frequency response are seen in Figure~\ref{fig:db} and \ref{fig:ph}.

\begin{figure}[H] \centering
\includegraphics[width=0.6\linewidth]{../sim/sim_5_db.pdf}
\caption{Magnitude in response to frequency changes.}
\label{fig:db}
\end{figure}

\begin{figure}[H] \centering
\includegraphics[width=0.6\linewidth]{../sim/sim_5_ph.pdf}
\caption{Phase in response to frequency changes.}
\label{fig:ph}
\end{figure}

%inserir comentario
$V_s$ and $V_6$ differ since $V_6$ is connected to a capacitor and therefore it will be the sum of a natural and a forced solution whilst $V_s$ is just a voltage source. % ver se é preciso comentar aqui qlq cena mais (comparar c outro trabalho)

\section{Comparing Octave and Ngspice}

\subsection{Operating Point Analysis for t$<$0}
\tab The difference between the results is minimal, as can be seen by comparing the table. The results are equal since the circuit is linear.

%inserir tabelas comparativas
\par

\subsection{Operating Point Analysis for t>0 (natural solution)}
The graphics are similar, as one would expect since the components are linear and therefore the circuit is linear.
%inserir graficos 
\par

\subsection{Operating Point Analysis for t>0 (forced solution)}

The graphics are similar, since the circuit is linear and we are using ideal voltage and current sources.

%inserir graficos

\subsection{Frequency response}

As one would expect the graphics are very similar and one can't find any significant discrepancies. This is the result of the circuit being linear
