\section{Octave and Ngspice Comparison}
\label{sec:comparision}

\subsection{Analysis for t$<$0}
\begin{table}[H]
	\centering
  \begin{tabular}{|l|r|}
    \hline    
    {\bf Name} & {\bf Value [V]} \\ \hline
	\input{../mat/volt_tab1}
  \end{tabular}
  \begin{tabular}{|l|r|}
    \hline    
    {\bf Name} & {\bf Value [A]} \\ \hline
	\input{../mat/current_tab1}
  \end{tabular}
  \begin{tabular}{|l|r|}
    \hline    
    {\bf Name} & {\bf Value [A or V]} \\ \hline
    @gib[i] & -2.60149e-04\\ \hline
@id[current] & 1.012649e-03\\ \hline
@rr1[i] & 2.482093e-04\\ \hline
@rr2[i] & -2.60149e-04\\ \hline
@rr3[i] & -1.19393e-05\\ \hline
@rr4[i] & 1.192184e-03\\ \hline
@rr5[i] & 1.272797e-03\\ \hline
@rr6[i] & 9.439751e-04\\ \hline
@rr7[i] & 9.439751e-04\\ \hline
v(1) & 7.053027e+00\\ \hline
v(2) & 6.795298e+00\\ \hline
v(3) & 6.264646e+00\\ \hline
v(4) & 1.949470e+00\\ \hline
v(5) & 6.831425e+00\\ \hline
v(6) & 1.080049e+01\\ \hline
v(7) & -9.54723e-01\\ \hline
v(9) & 0.000000e+00\\ \hline

  \end{tabular}
  \caption{Voltages and currents obtained by node analysis in Octave on the left and by operating point in Ngspice on the right.}
    \label{tab:comp1}
\end{table}

\subsection{Natural solution for t$>$0}
\begin{table}[H]
  \centering
  \begin{tabular}{|l|r|}
    \hline    
    {\bf Name} & {\bf Value [A], [V], [Ohm]} \\ \hline
	\input{../mat/point2_tab}
  \end{tabular}
  \begin{tabular}{|l|r|}
    \hline    
    {\bf Name} & {\bf Value [A or V]} \\ \hline
    \input{../sim/op_2_tab}
  \end{tabular}
  \caption{On the left, results from Octave, and on the right Ngspice. These tables give us the current and voltage between nodes 6 and 8. All the currents and voltages in other nodes/branches are null.}
  \label{tab:comp2}
\end{table}
