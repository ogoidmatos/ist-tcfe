\section{Simulation Analysis}
\label{sec:simulation}
In this section, we will show the results of the Ngspice
simulation. Here the results were obtained with the complex
Philips transistor model given by the Professor, which lead
to some differences when compared with the theoretical
analysis. This was expected since here We encounter a
circuit that is not linear, and, as a result, the
theoretical model used is only an approximation of the nonlinear circuit.
Below is presented the graph for the results obtained. In
Figure~\ref{fig:sim1} it can be observed the gain as a
function of the frequency in log10 scale.

%inserir grafico
\begin{figure}[H] \centering
\includegraphics[width=0.5\linewidth]{vo2f.eps}
\caption{Gain in frequency response analysis.}
\label{fig:sim1}
\end{figure}

The Table~\ref{tab:spice} presents the output voltage
gain in the passband, the lower and upper 3dB cut off
frequencies, as well as the input and output impedances in Table~\ref{tab:imped}.

\begin{table}[H]
 \centering
 \begin{tabular}{|l|r|}
 \hline
 {\bf Name} & {\bf Value} \\ \hline
\input{../sim/spice_results_tab}
 \end{tabular}
 \caption{Voltage gain, bandwidth and lower cutoff frequency.}
 \label{tab:spice}
 \end{table}
 
 \begin{table}[H]
 \centering
 \begin{tabular}{|l|r|}
 \hline
 {\bf Name} & {\bf Value} \\ \hline
\input{../sim/op_2_tab}
\input{../sim/op_in_tab}
 \end{tabular}
 \caption{Input and output impedances of the circuit.}
 \label{tab:imped}
 \end{table}

The Table~\ref{tab:forwardbias} presents for the NPN $V_{CE}$ and $V_{BE}$, as well as for
PNP $V_{EC}$ and $V_{EB}$. As it can be seen, $V_{CE}$ $>$ $V_{BE}$ and $V_{EC}$ $>$ $V_{EB}$, which allows us to
conclude that the transistors are in forward active region (FAR) as necessary.

\begin{table}[H]
 \centering
 \begin{tabular}{|l|r|}
 \hline
 {\bf Name} & {\bf Value [V]} \\ \hline
@gib[i] & -2.60149e-04\\ \hline
@id[current] & 1.012649e-03\\ \hline
@rr1[i] & 2.482093e-04\\ \hline
@rr2[i] & -2.60149e-04\\ \hline
@rr3[i] & -1.19393e-05\\ \hline
@rr4[i] & 1.192184e-03\\ \hline
@rr5[i] & 1.272797e-03\\ \hline
@rr6[i] & 9.439751e-04\\ \hline
@rr7[i] & 9.439751e-04\\ \hline
v(1) & 7.053027e+00\\ \hline
v(2) & 6.795298e+00\\ \hline
v(3) & 6.264646e+00\\ \hline
v(4) & 1.949470e+00\\ \hline
v(5) & 6.831425e+00\\ \hline
v(6) & 1.080049e+01\\ \hline
v(7) & -9.54723e-01\\ \hline
v(9) & 0.000000e+00\\ \hline

 \end{tabular}
 \caption{Verification of the FAR.}
 \label{tab:forwardbias}
 \end{table}

Using various values for the coupling capacitor, it became clear that it has a big
effect on the lower cutoff frequency. By increasing its value, we were able to get
closer to 20Hz, although increasing it has a respective increase in the overall
cost of our amplifier.
As far as the bypass capacitor is concerned, again by using multiple values for
the capacitance we were able to understand that its main purpose it to stabilize
the gain, allowing the gain to be stable in the passband that we desire. Here once
again a compromise had to be made due to the fact that by increasing the
capacitance the cost would also increasing.
The bypass capacitor will be an open circuit for low frequencies (where the DC is
dominant) and a short-circuit for higher frequencies where AC is dominant. As a
result, and as mentioned before the primary purpose of this capacitor is to
stabilize in the passband that is desired.
The $R_c$ resistor was also studied. The higher its resistance got, the lower the
output impedance was, and as a result, also here a study was as made to find the
optimal value for this resistor.
Finally, and as it can be seen in Table~\ref{tab:merit_spice}, our merit was excelent, resulting in great gain, bandwidth, and lower cutoff frequency. In order to achieve these, the cost of the circuit was high, but it was compensated by the high quality, which allowed us to achieve a very high merit, as seen in the table. 

\begin{table}[H]
 \centering
 \begin{tabular}{|l|r|}
 \hline
 {\bf Name} & {\bf Value} \\ \hline
\input{../sim/merito_tab}
 \end{tabular}
 \caption{Cost, quality and merit of the circuit.}
 \label{tab:merit_spice}
\end{table}
