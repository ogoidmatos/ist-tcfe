\section{Simulation Analysis}
\label{sec:simulation}
Since the voltage and current sources have constant values we only need to use the Operating Point analysis to fully study the circuit.

\subsection{Operating Point Analysis}

While using Ngspice we encountered some problems with the use of current-controlled voltage sources, to overcome them we introduced a 0V voltage source between the resistors 6 and 7.

Table~\ref{tab:op} shows the simulated operating point results for the circuit
under analysis. Compared to the theoretical analysis results, one notices the
following differences: describe and explain the differences.

First, we must take into consideration that the currents flowing through R1, R2, R3 and R4 are, respectively, equal to Ix, Iy, Iz and Iw.
Comparing the tables 1 and 2 with table 3, we see that the difference between the different voltages and currents is minimal. Different digits beggin to appear only on the fourth decimal case and these disparities can be attributed to distinct methods of calculation and floating point.\par
On the other hand, by analysing table 3, we note that in each resistor the current and voltage values have the same sign, as it should be since the power of a resistor must be positive. \par
% ver se de facto e a table 1 2 e 3 
Having verified that the theoretical results and the simulation results are equal, we say that the methods used to study the circuit are accurate.


\begin{table}[h]
  \centering
  \begin{tabular}{|l|r|}
    \hline    
    {\bf Name} & {\bf Value [A or V]} \\ \hline
    @gib[i] & -2.60149e-04\\ \hline
@id[current] & 1.012649e-03\\ \hline
@rr1[i] & 2.482093e-04\\ \hline
@rr2[i] & -2.60149e-04\\ \hline
@rr3[i] & -1.19393e-05\\ \hline
@rr4[i] & 1.192184e-03\\ \hline
@rr5[i] & 1.272797e-03\\ \hline
@rr6[i] & 9.439751e-04\\ \hline
@rr7[i] & 9.439751e-04\\ \hline
v(1) & 7.053027e+00\\ \hline
v(2) & 6.795298e+00\\ \hline
v(3) & 6.264646e+00\\ \hline
v(4) & 1.949470e+00\\ \hline
v(5) & 6.831425e+00\\ \hline
v(6) & 1.080049e+01\\ \hline
v(7) & -9.54723e-01\\ \hline
v(9) & 0.000000e+00\\ \hline

  \end{tabular}
  \caption{Operating point. A variable preceded by @ is of type {\em current}
    and expressed in Ampere; other variables are of type {\it voltage} and expressed in
    Volt.}
  \label{tab:op}
\end{table}





